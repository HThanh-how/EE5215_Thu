\documentclass[12pt,a4paper]{article}
\usepackage{fontspec}
\usepackage[vietnamese]{babel}
\usepackage{amsmath,amssymb}
\usepackage{siunitx}
\usepackage{graphicx}
\usepackage{geometry}
\usepackage{xcolor}
\usepackage{tikz}
\usetikzlibrary{shapes,arrows,positioning,calc}
\usepackage{enumitem}
\usepackage{booktabs}
\usepackage{fancyhdr}
\usepackage{hyperref}
\usepackage{tcolorbox}

\geometry{margin=2cm}
\setlength{\headheight}{15pt}
\addtolength{\topmargin}{-3pt}
\pagestyle{fancy}
\fancyhf{}
\fancyhead[L]{EE5215 - Câu 1: CPU Pipeline}
\fancyhead[R]{1.5 điểm}
\fancyfoot[C]{\thepage}

\definecolor{formulabox}{RGB}{230,240,255}
\definecolor{resultbox}{RGB}{220,255,220}
\definecolor{conceptbox}{RGB}{255,245,220}

\begin{document}

\begin{center}
\Large\textbf{CÂU HỎI 1: CPU PIPELINE}\\[0.3cm]
\large Learning Outcome: L.O.1 | Điểm: 1.5
\end{center}

\section{Đề bài}

Cho CPU với thời gian xử lý các tác vụ cho 1 lệnh (instruction) Load như sau:

\begin{center}
\begin{tabular}{|c|c|c|c|c|}
\hline
\textbf{Instruction Fetch} & \textbf{Instruction Decode} & \textbf{Execute} & \textbf{Memory Access} & \textbf{Write back} \\
\hline
180ps & 140ps & 160ps & 220ps & 120ps \\
\hline
\end{tabular}
\end{center}

Tính các thông số: Cycle time, Latency, Throughput, Speed-up trong 2 trường hợp:
\begin{enumerate}
\item Không sử dụng pipeline.
\item Sử dụng pipeline 5 tầng. Giả sử ở mỗi tầng pipeline phát sinh thêm thời gian trễ 15ps (do thanh ghi pipeline).
\end{enumerate}

\section{Kiến thức nền tảng}

\subsection{Pipeline là gì?}

\begin{tcolorbox}[colback=conceptbox,title=Khái niệm Pipeline]
\textbf{Pipeline (đường ống)} là kỹ thuật cho phép CPU thực hiện nhiều lệnh đồng thời bằng cách chia quá trình thực thi thành các giai đoạn (stages) và xử lý song song các lệnh ở các giai đoạn khác nhau.

\textbf{Ví dụ tương đồng:} Giống như dây chuyền sản xuất trong nhà máy - thay vì hoàn thành một sản phẩm rồi mới bắt đầu sản phẩm tiếp theo, nhiều sản phẩm được xử lý đồng thời ở các giai đoạn khác nhau.
\end{tcolorbox}

\subsection{5 giai đoạn của MIPS Pipeline}

\begin{center}
\begin{tikzpicture}[node distance=1.5cm, auto,
    stage/.style={rectangle, draw, fill=blue!20, text width=2.5cm, text centered, minimum height=1cm}]
    
    \node[stage] (IF) {IF\\Instruction Fetch};
    \node[stage, right=0.3cm of IF] (ID) {ID\\Instruction Decode};
    \node[stage, right=0.3cm of ID] (EX) {EX\\Execute};
    \node[stage, right=0.3cm of EX] (MEM) {MEM\\Memory Access};
    \node[stage, right=0.3cm of MEM] (WB) {WB\\Write Back};
    
    \draw[->,thick] (IF) -- (ID);
    \draw[->,thick] (ID) -- (EX);
    \draw[->,thick] (EX) -- (MEM);
    \draw[->,thick] (MEM) -- (WB);
\end{tikzpicture}
\end{center}

\begin{itemize}
    \item \textbf{IF (Instruction Fetch - 180ps):} Đọc lệnh từ bộ nhớ instruction cache
    \item \textbf{ID (Instruction Decode - 140ps):} Giải mã lệnh, đọc thanh ghi nguồn
    \item \textbf{EX (Execute - 160ps):} Thực hiện tính toán ALU hoặc tính địa chỉ
    \item \textbf{MEM (Memory Access - 220ps):} Đọc/ghi bộ nhớ data cache
    \item \textbf{WB (Write Back - 120ps):} Ghi kết quả về thanh ghi đích
\end{itemize}

\subsection{Các thông số quan trọng}

\begin{tcolorbox}[colback=formulabox,title=Định nghĩa các thông số]
\begin{enumerate}
    \item \textbf{Cycle time (Thời gian chu kỳ):}
    \begin{itemize}
        \item Non-pipeline: Thời gian hoàn thành 1 lệnh = tổng thời gian tất cả giai đoạn
        \item Pipeline: Thời gian của giai đoạn chậm nhất (+ overhead thanh ghi)
    \end{itemize}
    
    \item \textbf{Latency (Độ trễ):} Thời gian từ khi bắt đầu đến khi hoàn thành 1 lệnh
    
    \item \textbf{Throughput (Thông lượng):} Số lệnh hoàn thành trong một đơn vị thời gian
    \[
    \text{Throughput} = \frac{1}{\text{Cycle time}}
    \]
    
    \item \textbf{Speed-up (Tăng tốc):} Tỷ lệ cải thiện hiệu suất khi dùng pipeline
    \[
    \text{Speed-up} = \frac{\text{Throughput}_{\text{pipeline}}}{\text{Throughput}_{\text{non-pipeline}}}
    \]
\end{enumerate}
\end{tcolorbox}

\section{Lời giải chi tiết}

\subsection{Trường hợp 1: Không sử dụng Pipeline}

\begin{tcolorbox}[colback=white,title=Phân tích Non-Pipeline]
Khi KHÔNG dùng pipeline, CPU phải hoàn thành tất cả 5 giai đoạn của 1 lệnh trước khi bắt đầu lệnh tiếp theo.

\textbf{Timeline cho 1 lệnh:}
\begin{center}
\begin{tikzpicture}[scale=0.8]
    \draw[->] (0,0) -- (12,0) node[right] {Thời gian (ps)};
    \draw[fill=red!30] (0,0.5) rectangle (1.8,1) node[midway] {IF};
    \draw[fill=orange!30] (1.8,0.5) rectangle (3.2,1) node[midway] {ID};
    \draw[fill=yellow!30] (3.2,0.5) rectangle (4.8,1) node[midway] {EX};
    \draw[fill=green!30] (4.8,0.5) rectangle (7,1) node[midway] {MEM};
    \draw[fill=blue!30] (7,0.5) rectangle (8.2,1) node[midway] {WB};
    
    \foreach \x/\label in {0/0, 1.8/180, 3.2/320, 4.8/480, 7/700, 8.2/820} {
        \draw (\x,0.1) -- (\x,-0.1) node[below] {\label};
    }
\end{tikzpicture}
\end{center}
\end{tcolorbox}

\textbf{Bước 1: Tính Cycle time}

\begin{tcolorbox}[colback=formulabox]
\[
T_{cycle} = T_{IF} + T_{ID} + T_{EX} + T_{MEM} + T_{WB}
\]
\[
T_{cycle} = 180 + 140 + 160 + 220 + 120 = \mathbf{820\,ps}
\]
\end{tcolorbox}

\textbf{Bước 2: Tính Latency}

Vì không có pipeline, latency bằng cycle time:
\begin{tcolorbox}[colback=resultbox]
\[
\text{Latency} = T_{cycle} = \mathbf{820\,ps}
\]
\end{tcolorbox}

\textbf{Bước 3: Tính Throughput}

\begin{tcolorbox}[colback=formulabox]
\[
\text{Throughput} = \frac{1}{T_{cycle}} = \frac{1}{820 \times 10^{-12}\,s}
\]
\[
\text{Throughput} = 1.2195 \times 10^9 \text{ lệnh/s} = \mathbf{1.22\,GIPS}
\]
\end{tcolorbox}

\textbf{GIPS} = Giga Instructions Per Second (tỷ lệnh mỗi giây)

\subsection{Trường hợp 2: Sử dụng Pipeline 5 tầng}

\begin{tcolorbox}[colback=white,title=Phân tích Pipeline]
Khi dùng pipeline, các lệnh được xử lý song song ở các giai đoạn khác nhau.

\textbf{Quy tắc quan trọng:}
\begin{itemize}
    \item Cycle time = thời gian của giai đoạn \textbf{chậm nhất} + overhead
    \item Tất cả các giai đoạn phải hoạt động đồng bộ với cùng một clock
    \item Mỗi tầng có thanh ghi pipeline (pipeline register) để lưu dữ liệu
\end{itemize}
\end{tcolorbox}

\textbf{Bước 1: Xác định giai đoạn chậm nhất}

\begin{center}
\begin{tabular}{|c|c|c|c|c|c|}
\hline
Giai đoạn & IF & ID & EX & MEM & WB \\
\hline
Thời gian (ps) & 180 & 140 & 160 & \textcolor{red}{\textbf{220}} & 120 \\
\hline
\end{tabular}
\end{center}

$\Rightarrow$ Giai đoạn \textbf{MEM (Memory Access)} là chậm nhất với 220ps

\textbf{Bước 2: Tính Cycle time (có tính overhead)}

\begin{tcolorbox}[colback=formulabox]
\[
T_{cycle,pipeline} = \max(T_{IF}, T_{ID}, T_{EX}, T_{MEM}, T_{WB}) + T_{overhead}
\]
\[
T_{cycle,pipeline} = \max(180, 140, 160, 220, 120) + 15 = 220 + 15 = \mathbf{235\,ps}
\]
\end{tcolorbox}

\textbf{Tại sao cần thêm overhead 15ps?}
\begin{itemize}
    \item Thanh ghi pipeline (flip-flops) cần thời gian để latch dữ liệu
    \item Setup time và propagation delay của flip-flops
    \item Clock skew compensation
\end{itemize}

\textbf{Bước 3: Tính Latency}

Latency = thời gian để 1 lệnh đi qua tất cả 5 giai đoạn:

\begin{tcolorbox}[colback=formulabox]
\[
\text{Latency} = n \times T_{cycle,pipeline} = 5 \times 235 = \mathbf{1175\,ps}
\]
\end{tcolorbox}

\textbf{Chú ý:} Latency của pipeline \textbf{lớn hơn} non-pipeline (1175ps > 820ps) do overhead từ thanh ghi pipeline ở mỗi tầng!

\textbf{Bước 4: Tính Throughput (khi pipeline đầy)}

\begin{tcolorbox}[colback=formulabox]
\[
\text{Throughput} = \frac{1}{T_{cycle,pipeline}} = \frac{1}{235 \times 10^{-12}\,s}
\]
\[
\text{Throughput} = 4.255 \times 10^9 \text{ lệnh/s} = \mathbf{4.26\,GIPS}
\]
\end{tcolorbox}

\textbf{Timeline của Pipeline (với 5 lệnh):}

\begin{center}
\begin{tikzpicture}[scale=0.6]
    \draw[->] (0,0) -- (12,0) node[right] {Chu kỳ};
    \draw[->] (0,0) -- (0,6) node[above] {Lệnh};
    
    % Labels
    \foreach \y/\label in {1/I1, 2/I2, 3/I3, 4/I4, 5/I5} {
        \node at (-0.5,\y) {\label};
    }
    \foreach \x in {1,...,9} {
        \node at (\x,-0.3) {\x};
    }
    
    % Instruction 1
    \draw[fill=red!30] (1,1) rectangle (2,1.5); \node at (1.5,1.25) {\tiny IF};
    \draw[fill=orange!30] (2,1) rectangle (3,1.5); \node at (2.5,1.25) {\tiny ID};
    \draw[fill=yellow!30] (3,1) rectangle (4,1.5); \node at (3.5,1.25) {\tiny EX};
    \draw[fill=green!30] (4,1) rectangle (5,1.5); \node at (4.5,1.25) {\tiny MEM};
    \draw[fill=blue!30] (5,1) rectangle (6,1.5); \node at (5.5,1.25) {\tiny WB};
    
    % Instruction 2
    \draw[fill=red!30] (2,2) rectangle (3,2.5);
    \draw[fill=orange!30] (3,2) rectangle (4,2.5);
    \draw[fill=yellow!30] (4,2) rectangle (5,2.5);
    \draw[fill=green!30] (5,2) rectangle (6,2.5);
    \draw[fill=blue!30] (6,2) rectangle (7,2.5);
    
    % Instruction 3
    \draw[fill=red!30] (3,3) rectangle (4,3.5);
    \draw[fill=orange!30] (4,3) rectangle (5,3.5);
    \draw[fill=yellow!30] (5,3) rectangle (6,3.5);
    \draw[fill=green!30] (6,3) rectangle (7,3.5);
    \draw[fill=blue!30] (7,3) rectangle (8,3.5);
    
    % Instruction 4
    \draw[fill=red!30] (4,4) rectangle (5,4.5);
    \draw[fill=orange!30] (5,4) rectangle (6,4.5);
    \draw[fill=yellow!30] (6,4) rectangle (7,4.5);
    \draw[fill=green!30] (7,4) rectangle (8,4.5);
    \draw[fill=blue!30] (8,4) rectangle (9,4.5);
    
    % Instruction 5
    \draw[fill=red!30] (5,5) rectangle (6,5.5);
    \draw[fill=orange!30] (6,5) rectangle (7,5.5);
    \draw[fill=yellow!30] (7,5) rectangle (8,5.5);
    \draw[fill=green!30] (8,5) rectangle (9,5.5);
    \draw[fill=blue!30] (9,5) rectangle (10,5.5);
\end{tikzpicture}
\end{center}

\textbf{Bước 5: Tính Speed-up}

\begin{tcolorbox}[colback=resultbox,title=Kết quả Speed-up]
\textbf{Cách 1: Dựa trên Throughput}
\[
\text{Speed-up} = \frac{\text{Throughput}_{\text{pipeline}}}{\text{Throughput}_{\text{non-pipeline}}} = \frac{4.26}{1.22} = \mathbf{3.49}
\]

\textbf{Cách 2: Dựa trên Cycle time}
\[
\text{Speed-up} = \frac{T_{cycle,\text{non-pipeline}}}{T_{cycle,\text{pipeline}}} = \frac{820}{235} = \mathbf{3.49}
\]
\end{tcolorbox}

\section{Tổng kết kết quả}

\begin{center}
\begin{tabular}{|l|c|c|}
\hline
\textbf{Thông số} & \textbf{Non-Pipeline} & \textbf{Pipeline 5 tầng} \\
\hline
Cycle time & 820 ps & 235 ps \\
\hline
Latency & 820 ps & 1175 ps \\
\hline
Throughput & 1.22 GIPS & 4.26 GIPS \\
\hline
Speed-up & 1× (baseline) & 3.49× \\
\hline
\end{tabular}
\end{center}

\section{Phân tích thêm}

\subsection{Tại sao Speed-up không đạt lý tưởng 5×?}

\begin{tcolorbox}[colback=conceptbox]
\textbf{Speed-up lý tưởng} với n giai đoạn là n×. Ở đây với 5 tầng, lý tưởng là 5×.

\textbf{Tại sao chỉ đạt 3.49×?}
\begin{enumerate}
    \item \textbf{Các giai đoạn không cân bằng:} MEM = 220ps, WB = 120ps $\Rightarrow$ lãng phí thời gian ở các giai đoạn nhanh hơn
    \item \textbf{Pipeline overhead:} 15ps × 5 tầng = 75ps thêm vào tổng latency
    \item \textbf{Công thức thực tế:}
    \[
    \text{Speed-up}_{\text{max}} = \frac{\sum T_i}{\max(T_i) + T_{overhead}} = \frac{820}{235} = 3.49
    \]
\end{enumerate}
\end{tcolorbox}

\subsection{Làm sao để cải thiện Speed-up?}

\begin{enumerate}
    \item \textbf{Cân bằng các giai đoạn:} Chia nhỏ giai đoạn MEM hoặc gộp các giai đoạn ngắn
    \item \textbf{Giảm overhead:} Sử dụng flip-flops nhanh hơn
    \item \textbf{Super-pipelining:} Chia thành nhiều giai đoạn nhỏ hơn
\end{enumerate}

\end{document}
