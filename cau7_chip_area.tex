\documentclass[12pt,a4paper]{article}
\usepackage{fontspec}
\usepackage[vietnamese]{babel}
\usepackage{amsmath,amssymb}
\usepackage{siunitx}
\usepackage{geometry}
\usepackage{xcolor}
\usepackage{tikz}
\usepackage{enumitem}
\usepackage{booktabs}
\usepackage{fancyhdr}
\usepackage{tcolorbox}
\usepackage{multirow}

\geometry{margin=2cm}
\pagestyle{fancy}
\fancyhf{}
\fancyhead[L]{EE5215 - Câu 7: Chip Area Calculation}
\fancyhead[R]{1.0 điểm}
\fancyfoot[C]{\thepage}

\definecolor{formulabox}{RGB}{230,240,255}
\definecolor{resultbox}{RGB}{220,255,220}
\definecolor{conceptbox}{RGB}{255,245,220}

\begin{document}

\begin{center}
\Large\textbf{CÂU HỎI 7: TÍNH DIỆN TÍCH CHIP}\\[0.3cm]
\large Learning Outcome: L.O.6 | Điểm: 1.0
\end{center}

\section{Đề bài}

Phân tích hệ thống SoC được thiết kế với công nghệ 65nm, bao gồm các thành phần trên 1 die như sau:

\begin{center}
\begin{tabular}{|l|c|}
\hline
\textbf{Thành phần} & \textbf{Diện tích (Đơn vị: A)} \\
\hline
CPU core & 750 \\
\hline
GPU core & 300 \\
\hline
DMA controller & 100 \\
\hline
Peripheral I/O (UART, SPI, I2C) & 350 \\
\hline
Bus và control unit & 500 \\
\hline
Internal SRAM (1 MB) & 4096 \\
\hline
L2 cache (256 KB) & 1024 \\
\hline
\end{tabular}
\end{center}

\textbf{Công thức:} $1\,A = 1481\,\text{rbe} = 10^6 \times f^2$ ($f$: feature size in $\mu$m)

Biết rằng 15\% diện tích chip dành cho IO pads và diện tích khung cần thiết. Tính diện tích chip khi chưa đóng gói (gross chip area) theo đơn vị mm².

\section{Kiến thức nền tảng}

\subsection{Các đơn vị đo diện tích chip}

\begin{tcolorbox}[colback=conceptbox,title=Đơn vị diện tích trong thiết kế chip]
\textbf{1. rbe (relative bit equivalent):}
\begin{itemize}
    \item Đơn vị chuẩn hóa để so sánh diện tích
    \item 1 rbe = diện tích tương đương lưu 1 bit (SRAM cell đơn giản)
\end{itemize}

\textbf{2. A (Area Unit):}
\begin{itemize}
    \item Đơn vị quy ước theo công nghệ
    \item $1\,A = 1481\,\text{rbe}$
    \item Phụ thuộc vào feature size: $A \propto f^2$
\end{itemize}

\textbf{3. Đơn vị vật lý:}
\begin{itemize}
    \item $\mu m^2$ (micromet vuông)
    \item $mm^2$ (milimet vuông) - thường dùng cho chip
    \item $1\,mm^2 = 10^6\,\mu m^2$
\end{itemize}
\end{tcolorbox}

\subsection{Cấu trúc diện tích chip}

\begin{tcolorbox}[colback=formulabox]
\textbf{Gross Chip Area} = Core Area + IO Pad Area + Seal Ring

\begin{center}
\begin{tikzpicture}[scale=0.7]
    % Outer rectangle (Gross Area)
    \draw[thick] (0,0) rectangle (8,6);
    \node at (4,6.3) {Gross Chip Area};
    
    % IO Pad ring
    \draw[fill=gray!30] (0.5,0.5) rectangle (7.5,5.5);
    \node at (4,0.25) {\tiny IO Pads};
    
    % Core area
    \draw[fill=blue!20] (1,1) rectangle (7,5);
    \node at (4,3) {Core Area};
    \node at (4,2.5) {\tiny (Logic + Memory)};
    
    % Labels
    \draw[<->] (8.3,0) -- (8.3,6);
    \node at (8.6,3) {\rotatebox{90}{Gross}};
    
    \draw[<->] (7.2,1) -- (7.2,5);
    \node at (6.9,3) {\rotatebox{90}{\tiny Core}};
\end{tikzpicture}
\end{center}

\textbf{Công thức:}
\[
\text{Core Area} = (1 - \text{IO ratio}) \times \text{Gross Area}
\]
\[
\text{Gross Area} = \frac{\text{Core Area}}{1 - \text{IO ratio}}
\]
\end{tcolorbox}

\subsection{Công nghệ 65nm}

\begin{tcolorbox}[colback=conceptbox]
\textbf{Feature size (f) = 65nm:}
\begin{itemize}
    \item Minimum gate length = 65nm = 0.065 $\mu$m = 0.000065 mm
    \item Đây là node công nghệ khá phổ biến (2005-2010)
    \item Transistor density: $\sim$2-3 million gates/mm²
\end{itemize}

\textbf{Ý nghĩa của công thức $A = 10^6 \times f^2$:}
\begin{itemize}
    \item Diện tích scale theo bình phương feature size
    \item Khi công nghệ nhỏ hơn (shrink), diện tích giảm theo $f^2$
\end{itemize}
\end{tcolorbox}

\section{Lời giải chi tiết}

\subsection{Bước 1: Tính tổng diện tích các thành phần (Core Area) theo đơn vị A}

\begin{tcolorbox}[colback=formulabox,title=Cộng diện tích tất cả components]
\begin{center}
\begin{tabular}{|l|r|l|}
\hline
\textbf{Thành phần} & \textbf{Diện tích (A)} & \textbf{Ghi chú} \\
\hline
CPU core & 750 & Logic processing \\
\hline
GPU core & 300 & Graphics processing \\
\hline
DMA controller & 100 & Data transfer \\
\hline
Peripheral I/O & 350 & UART, SPI, I2C \\
\hline
Bus \& control & 500 & Interconnect \\
\hline
SRAM (1 MB) & 4096 & Internal memory \\
\hline
L2 cache (256 KB) & 1024 & Cache memory \\
\hline
\hline
\textbf{Tổng Core Area} & \textbf{7120 A} & \\
\hline
\end{tabular}
\end{center}

\[
A_{core} = 750 + 300 + 100 + 350 + 500 + 4096 + 1024 = \mathbf{7120\,A}
\]
\end{tcolorbox}

\subsection{Bước 2: Tính giá trị của 1 A theo đơn vị vật lý}

Theo đề bài: $1\,A = 10^6 \times f^2$ với $f$ tính bằng $\mu$m

\begin{tcolorbox}[colback=formulabox,title=Chuyển đổi đơn vị]
\textbf{Feature size:}
\[
f = 65\,nm = 65 \times 10^{-3}\,\mu m = 0.065\,\mu m
\]

\textbf{Tính 1 A theo $\mu m^2$:}
\[
1\,A = 10^6 \times f^2 = 10^6 \times (0.065)^2\,\mu m^2
\]
\[
1\,A = 10^6 \times 0.004225\,\mu m^2 = 4225\,\mu m^2
\]

\textbf{Chuyển sang $mm^2$:}
\[
1\,\mu m = 10^{-3}\,mm \Rightarrow 1\,\mu m^2 = 10^{-6}\,mm^2
\]
\[
1\,A = 4225 \times 10^{-6}\,mm^2 = \mathbf{0.004225\,mm^2}
\]
\end{tcolorbox}

\subsection{Bước 3: Tính Core Area theo mm²}

\begin{tcolorbox}[colback=formulabox]
\[
\text{Core Area} = A_{core} \times (\text{1 A in } mm^2)
\]
\[
\text{Core Area} = 7120 \times 0.004225\,mm^2
\]
\[
\text{Core Area} = \mathbf{30.082\,mm^2}
\]
\end{tcolorbox}

\subsection{Bước 4: Tính Gross Chip Area}

\begin{tcolorbox}[colback=resultbox,title=Tính diện tích chip chưa đóng gói]
Theo đề bài: 15\% diện tích dành cho IO pads và khung

$\Rightarrow$ Core Area chiếm 85\% Gross Area

\[
\text{Core Area} = 85\% \times \text{Gross Area}
\]
\[
\text{Gross Area} = \frac{\text{Core Area}}{0.85}
\]
\[
\text{Gross Area} = \frac{30.082}{0.85} = \boxed{\mathbf{35.39\,mm^2}}
\]
\end{tcolorbox}

\subsection{Kiểm tra kết quả}

\begin{tcolorbox}[colback=white,title=Verification]
\textbf{Phân bổ diện tích:}
\begin{itemize}
    \item Core Area: $35.39 \times 0.85 = 30.08\,mm^2$ ✓
    \item IO Pads + Frame: $35.39 \times 0.15 = 5.31\,mm^2$
    \item Tổng: $30.08 + 5.31 = 35.39\,mm^2$ ✓
\end{itemize}

\textbf{Kích thước tương đương:}
\[
\text{Die size} \approx \sqrt{35.39} \approx 5.95\,mm \times 5.95\,mm
\]
Hoặc dạng chữ nhật: khoảng $6\,mm \times 6\,mm$
\end{tcolorbox}

\section{Phân tích thêm}

\subsection{Phân bổ diện tích theo thành phần}

\begin{center}
\begin{tikzpicture}[scale=0.8]
    \pie[
        text=legend,
        radius=3,
        color={red!60, orange!60, yellow!60, green!60, blue!60, purple!60, cyan!60}
    ]{
        10.5/CPU (750A),
        4.2/GPU (300A),
        1.4/DMA (100A),
        4.9/Peripheral (350A),
        7.0/Bus (500A),
        57.5/SRAM (4096A),
        14.4/L2 Cache (1024A)
    }
\end{tikzpicture}
\end{center}

\begin{tcolorbox}[colback=conceptbox,title=Nhận xét]
\textbf{Phân tích phân bổ diện tích:}
\begin{itemize}
    \item \textbf{Memory chiếm chủ đạo:} SRAM + L2 cache = 5120A = \textbf{71.9\%} core area
    \item \textbf{Logic (CPU+GPU):} 1050A = \textbf{14.7\%}
    \item \textbf{Peripherals + Bus:} 950A = \textbf{13.3\%}
\end{itemize}

\textbf{Đây là đặc điểm điển hình của SoC:}
\begin{itemize}
    \item Memory-dominated design
    \item SRAM density quyết định phần lớn diện tích
    \item Logic scale tốt hơn memory khi shrink technology
\end{itemize}
\end{tcolorbox}

\subsection{So sánh với các technology nodes khác}

\begin{tcolorbox}[colback=formulabox]
Giả sử cùng thiết kế 7120A, với các technology nodes khác:

\begin{center}
\begin{tabular}{|c|c|c|c|}
\hline
\textbf{Node} & \textbf{f ($\mu$m)} & \textbf{1 A ($\mu m^2$)} & \textbf{Core Area (mm²)} \\
\hline
180nm & 0.18 & 32400 & 230.7 \\
\hline
130nm & 0.13 & 16900 & 120.3 \\
\hline
90nm & 0.09 & 8100 & 57.7 \\
\hline
\textbf{65nm} & \textbf{0.065} & \textbf{4225} & \textbf{30.08} \\
\hline
45nm & 0.045 & 2025 & 14.4 \\
\hline
28nm & 0.028 & 784 & 5.6 \\
\hline
\end{tabular}
\end{center}

\textbf{Quy luật:} Diện tích giảm theo $f^2$ khi shrink technology.
\end{tcolorbox}

\section{Tổng kết}

\begin{tcolorbox}[colback=resultbox,title=Kết quả cuối cùng]
\textbf{Cho SoC với công nghệ 65nm:}
\begin{itemize}
    \item Tổng Core Area: 7120 A = \textbf{30.08 mm²}
    \item IO Pads + Frame (15\%): 5.31 mm²
    \item \textbf{Gross Chip Area: 35.39 mm²}
\end{itemize}

\textbf{Công thức tổng quát:}
\[
\boxed{\text{Gross Area} = \frac{\sum A_i \times 10^6 \times f^2}{(1-\text{IO ratio}) \times 10^6}}
\]
\end{tcolorbox}

\end{document}
