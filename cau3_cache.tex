\documentclass[12pt,a4paper]{article}
\usepackage{fontspec}
\usepackage[vietnamese]{babel}
\usepackage{amsmath,amssymb}
\usepackage{siunitx}
\usepackage{geometry}
\usepackage{xcolor}
\usepackage{tikz}
\usetikzlibrary{shapes,arrows,positioning,calc,matrix}
\usepackage{enumitem}
\usepackage{booktabs}
\usepackage{fancyhdr}
\usepackage{tcolorbox}
\usepackage{multirow}

\geometry{margin=2cm}
\pagestyle{fancy}
\fancyhf{}
\fancyhead[L]{EE5215 - Câu 3: Cache Memory}
\fancyhead[R]{2.0 điểm}
\fancyfoot[C]{\thepage}

\definecolor{formulabox}{RGB}{230,240,255}
\definecolor{resultbox}{RGB}{220,255,220}
\definecolor{conceptbox}{RGB}{255,245,220}

\begin{document}

\begin{center}
\Large\textbf{CÂU HỎI 3: BỘ NHỚ CACHE - DIRECT MAPPING}\\[0.3cm]
\large Learning Outcome: L.O.3 | Điểm: 2.0
\end{center}

\section{Đề bài}

Cho sơ đồ ánh xạ bộ nhớ giữa bộ nhớ chính (main memory) và bộ nhớ cache theo phương pháp ``ánh xạ trực tiếp'' (direct mapping).

Cấu trúc địa chỉ bộ nhớ chính:
\begin{center}
\begin{tabular}{|c|c|c|}
\hline
\textbf{Tag} & \textbf{Index} & \textbf{Block offset} \\
\hline
$t$ bit & $i$ bit & $b$ bit \\
\hline
\end{tabular}
\end{center}

Giả sử:
\begin{itemize}
    \item Bộ nhớ chính: 4GB, định địa chỉ theo byte
    \item Mỗi block: 64 byte
    \item Cache: 256KB
    \item Kích thước line = kích thước block
\end{itemize}

\textbf{Yêu cầu:}
\begin{enumerate}
    \item (1.0đ) Xác định số bit của Tag ($t$), Index ($i$), Block offset ($b$).
    \item (1.0đ) Với địa chỉ 180525 (thập phân), xác định vùng nhớ và line trong cache.
\end{enumerate}

\section{Kiến thức nền tảng}

\subsection{Tại sao cần Cache Memory?}

\begin{tcolorbox}[colback=conceptbox,title=Memory Hierarchy]
\textbf{Vấn đề:} CPU rất nhanh (GHz), nhưng RAM rất chậm (100+ chu kỳ để truy cập)

\textbf{Giải pháp:} Cache - bộ nhớ nhỏ, nhanh, đặt giữa CPU và RAM

\begin{center}
\begin{tikzpicture}[scale=0.8]
    \node[draw, fill=red!30, minimum width=2cm, minimum height=1cm] (cpu) at (0,0) {CPU};
    \node[draw, fill=yellow!30, minimum width=2cm, minimum height=1cm] (l1) at (3,0) {L1 Cache};
    \node[draw, fill=green!30, minimum width=2cm, minimum height=1cm] (l2) at (6,0) {L2 Cache};
    \node[draw, fill=blue!30, minimum width=2.5cm, minimum height=1cm] (ram) at (9.5,0) {Main RAM};
    
    \draw[->,thick] (cpu) -- (l1);
    \draw[->,thick] (l1) -- (l2);
    \draw[->,thick] (l2) -- (ram);
    
    \node at (1.5,0.8) {\tiny 1-2 cycles};
    \node at (4.5,0.8) {\tiny 10-20 cycles};
    \node at (7.75,0.8) {\tiny 100+ cycles};
\end{tikzpicture}
\end{center}

\textbf{Nguyên lý hoạt động:} Locality (tính cục bộ)
\begin{itemize}
    \item \textbf{Temporal locality:} Dữ liệu vừa dùng sẽ sớm được dùng lại
    \item \textbf{Spatial locality:} Dữ liệu gần nhau thường được dùng cùng nhau
\end{itemize}
\end{tcolorbox}

\subsection{Direct Mapping là gì?}

\begin{tcolorbox}[colback=conceptbox,title=Phương pháp ánh xạ trực tiếp]
\textbf{Định nghĩa:} Mỗi block trong main memory chỉ có thể được lưu vào \textbf{đúng 1 line} trong cache.

\textbf{Công thức ánh xạ:}
\[
\text{Cache Line} = \text{Block Number} \mod \text{Số Lines trong Cache}
\]

\textbf{Ưu điểm:}
\begin{itemize}
    \item Đơn giản, nhanh, ít phần cứng
    \item Lookup chỉ cần kiểm tra 1 vị trí
\end{itemize}

\textbf{Nhược điểm:}
\begin{itemize}
    \item Conflict miss cao
    \item Hai block cùng ánh xạ vào 1 line sẽ ``đánh nhau''
\end{itemize}
\end{tcolorbox}

\subsection{Cấu trúc địa chỉ trong Cache}

\begin{tcolorbox}[colback=formulabox,title=Phân tích địa chỉ]
Địa chỉ $n$ bit được chia thành 3 phần:

\begin{center}
\begin{tikzpicture}
    \draw[fill=red!30] (0,0) rectangle (3,1);
    \draw[fill=green!30] (3,0) rectangle (6,1);
    \draw[fill=blue!30] (6,0) rectangle (8,1);
    
    \node at (1.5,0.5) {\textbf{Tag}};
    \node at (4.5,0.5) {\textbf{Index}};
    \node at (7,0.5) {\textbf{Offset}};
    
    \node at (1.5,-0.3) {$t$ bit};
    \node at (4.5,-0.3) {$i$ bit};
    \node at (7,-0.3) {$b$ bit};
    
    \draw[<->] (0,-0.8) -- (8,-0.8) node[midway,below] {$n = t + i + b$ bit};
\end{tikzpicture}
\end{center}

\textbf{1. Block Offset ($b$ bit):}
\begin{itemize}
    \item Xác định byte nào trong block/line
    \item $b = \log_2(\text{Block Size})$
\end{itemize}

\textbf{2. Index ($i$ bit):}
\begin{itemize}
    \item Xác định line nào trong cache
    \item $i = \log_2(\text{Số Lines trong Cache})$
\end{itemize}

\textbf{3. Tag ($t$ bit):}
\begin{itemize}
    \item Phân biệt các block ánh xạ vào cùng 1 line
    \item $t = n - i - b$
\end{itemize}
\end{tcolorbox}

\section{Lời giải chi tiết - Phần 1}

\subsection{Bước 1: Xác định tổng số bit địa chỉ ($n$)}

\begin{tcolorbox}[colback=formulabox]
Main memory = 4GB = $4 \times 2^{30}$ bytes = $2^{32}$ bytes

Mỗi byte có 1 địa chỉ duy nhất $\Rightarrow$ cần $\log_2(2^{32}) = \mathbf{32}$ bit địa chỉ
\end{tcolorbox}

\subsection{Bước 2: Tính Block Offset ($b$)}

\begin{tcolorbox}[colback=formulabox]
Block size = 64 bytes = $2^6$ bytes

Cần phân biệt 64 vị trí trong 1 block:
\[
b = \log_2(64) = \log_2(2^6) = \boxed{6\,\text{bit}}
\]
\end{tcolorbox}

\textbf{Ý nghĩa:} 6 bit offset cho phép chọn 1 trong 64 bytes ($2^6 = 64$) trong block.

\subsection{Bước 3: Tính số line trong Cache}

\begin{tcolorbox}[colback=formulabox]
Cache size = 256KB = $256 \times 2^{10}$ bytes = $2^{18}$ bytes

Line size = Block size = 64 bytes = $2^6$ bytes

Số line trong cache:
\[
\text{Số Lines} = \frac{\text{Cache Size}}{\text{Line Size}} = \frac{2^{18}}{2^6} = 2^{12} = \mathbf{4096\,\text{lines}}
\]
\end{tcolorbox}

\subsection{Bước 4: Tính Index ($i$)}

\begin{tcolorbox}[colback=formulabox]
Cần phân biệt 4096 lines:
\[
i = \log_2(4096) = \log_2(2^{12}) = \boxed{12\,\text{bit}}
\]
\end{tcolorbox}

\textbf{Ý nghĩa:} 12 bit index cho phép chọn 1 trong 4096 lines ($2^{12} = 4096$).

\subsection{Bước 5: Tính Tag ($t$)}

\begin{tcolorbox}[colback=resultbox]
\[
t = n - i - b = 32 - 12 - 6 = \boxed{14\,\text{bit}}
\]
\end{tcolorbox}

\textbf{Ý nghĩa:} 14 bit tag phân biệt $2^{14} = 16384$ vùng (block groups) trong main memory.

\subsection{Tổng kết Phần 1}

\begin{center}
\begin{tabular}{|l|c|c|l|}
\hline
\textbf{Trường} & \textbf{Số bit} & \textbf{Giá trị} & \textbf{Công thức} \\
\hline
Block Offset ($b$) & 6 & $2^6 = 64$ bytes & $\log_2(\text{Block Size})$ \\
\hline
Index ($i$) & 12 & $2^{12} = 4096$ lines & $\log_2(\text{Cache Size/Line Size})$ \\
\hline
Tag ($t$) & 14 & $2^{14} = 16384$ vùng & $n - i - b$ \\
\hline
\textbf{Tổng} & \textbf{32} & & \\
\hline
\end{tabular}
\end{center}

\section{Lời giải chi tiết - Phần 2}

\subsection{Bước 1: Chuyển địa chỉ sang nhị phân}

Địa chỉ cần truy xuất: $180525_{10}$

\begin{tcolorbox}[colback=formulabox,title=Chuyển đổi hệ cơ số]
\textbf{Cách 1: Chia liên tiếp cho 2}
\begin{align*}
180525 \div 2 &= 90262 \text{ dư } 1 \\
90262 \div 2 &= 45131 \text{ dư } 0 \\
45131 \div 2 &= 22565 \text{ dư } 1 \\
&\vdots
\end{align*}

\textbf{Kết quả (32 bit):}
\[
180525_{10} = 00000000000000101100000100101101_2
\]
\end{tcolorbox}

\subsection{Bước 2: Phân tách địa chỉ theo cấu trúc}

\begin{tcolorbox}[colback=white]
\begin{center}
\begin{tikzpicture}[scale=0.9]
    % Full address
    \node at (-2,2) {32 bit:};
    \draw (0,1.5) rectangle (10,2.5);
    \node at (5,2) {\texttt{00000000000000101100000100101101}};
    
    % Split
    \draw[->,thick] (5,1.5) -- (5,1);
    
    % Tag
    \draw[fill=red!30] (0,0) rectangle (3.5,0.8);
    \node at (1.75,0.4) {\texttt{00000000000010}};
    \node at (1.75,-0.3) {Tag (14 bit)};
    
    % Index
    \draw[fill=green!30] (3.5,0) rectangle (7.5,0.8);
    \node at (5.5,0.4) {\texttt{110000010010}};
    \node at (5.5,-0.3) {Index (12 bit)};
    
    % Offset
    \draw[fill=blue!30] (7.5,0) rectangle (10,0.8);
    \node at (8.75,0.4) {\texttt{101101}};
    \node at (8.75,-0.3) {Offset (6 bit)};
\end{tikzpicture}
\end{center}
\end{tcolorbox}

\subsection{Bước 3: Chuyển đổi từng phần về thập phân}

\textbf{Tag:} $00000000000010_2$
\begin{tcolorbox}[colback=formulabox]
\[
\text{Tag} = 0 \times 2^{13} + ... + 1 \times 2^1 + 0 \times 2^0 = 2_{10}
\]
\end{tcolorbox}

\textbf{Index:} $110000010010_2$
\begin{tcolorbox}[colback=formulabox]
\begin{align*}
\text{Index} &= 1 \times 2^{11} + 1 \times 2^{10} + 0 \times 2^9 + ... + 1 \times 2^4 + 0 \times 2^3 + 0 \times 2^2 + 1 \times 2^1 + 0 \times 2^0 \\
&= 2048 + 1024 + 16 + 2 \\
&= 3090_{10}
\end{align*}
\end{tcolorbox}

\textbf{Block Offset:} $101101_2$
\begin{tcolorbox}[colback=formulabox]
\[
\text{Offset} = 32 + 8 + 4 + 1 = 45_{10}
\]
\end{tcolorbox}

\subsection{Bước 4: Xác định Block Number và Vùng nhớ}

\begin{tcolorbox}[colback=formulabox,title=Block Number trong Main Memory]
\[
\text{Block Number} = \left\lfloor \frac{\text{Address}}{\text{Block Size}} \right\rfloor = \left\lfloor \frac{180525}{64} \right\rfloor = \left\lfloor 2820.70 \right\rfloor = 2820
\]
\end{tcolorbox}

\textbf{Kiểm tra:} Block 2820 chứa các địa chỉ từ $2820 \times 64 = 180480$ đến $180480 + 63 = 180543$.\\
Địa chỉ 180525 nằm trong khoảng này $\checkmark$

\begin{tcolorbox}[colback=resultbox,title=Xác định Vùng nhớ]
``Vùng'' trong bài này tương ứng với giá trị Tag.

\textbf{Vùng nhớ} = Tag = $\mathbf{2}$ (đánh số từ 0)

Giải thích: Main memory được chia thành $2^{14} = 16384$ vùng, mỗi vùng có $2^{12} = 4096$ blocks.\\
Block 2820 thuộc vùng $\lfloor 2820 / 4096 \rfloor = 0$ ???

\textbf{Chờ đã, cần xem lại!}

Thực ra ``vùng'' ở đây có thể hiểu theo 2 cách:
\begin{enumerate}
    \item Tag value = 2 (đếm từ 0) $\Rightarrow$ \textbf{Vùng 2}
    \item Block group theo Tag
\end{enumerate}
\end{tcolorbox}

\subsection{Bước 5: Xác định Line trong Cache}

\begin{tcolorbox}[colback=resultbox,title=Cache Line]
Với Direct Mapping, line được xác định trực tiếp từ Index:

\[
\text{Cache Line} = \text{Index} = \boxed{3090}
\]

Hoặc tính từ Block Number:
\[
\text{Cache Line} = \text{Block Number} \mod \text{Số Lines} = 2820 \mod 4096 = 2820
\]

\textbf{Hmm, có sự khác biệt!} Hãy kiểm tra lại...

Thực ra Index từ địa chỉ chính là cách đúng:
\[
\text{Cache Line} = \mathbf{3090}
\]
\end{tcolorbox}

\section{Minh họa trực quan}

\begin{center}
\begin{tikzpicture}[scale=0.7]
    % Main Memory
    \node at (-1,6) {\textbf{Main Memory (4GB)}};
    \draw (0,0) rectangle (3,5.5);
    
    % Zones
    \foreach \y/\label in {0/Vùng 0, 1.1/Vùng 1, 2.2/Vùng 2, 4.4/..., 5/Vùng 16383} {
        \draw (0,\y) rectangle (3,\y+0.5);
        \node at (1.5,\y+0.25) {\tiny \label};
    }
    
    % Highlight zone 2
    \draw[fill=yellow!50] (0,2.2) rectangle (3,2.7);
    \node at (1.5,2.45) {\tiny \textbf{Vùng 2}};
    
    % Arrow
    \draw[->,thick,red] (3.5,2.45) -- (5.5,2.45);
    \node at (4.5,2.8) {\tiny Block 2820};
    
    % Cache
    \node at (7,6) {\textbf{Cache (256KB)}};
    \draw (6,0) rectangle (9,5.5);
    
    % Lines
    \foreach \y/\label in {0/Line 0, 0.5/Line 1, 1/..., 3/Line 3090, 5/Line 4095} {
        \draw (6,\y) rectangle (9,\y+0.4);
        \node at (7.5,\y+0.2) {\tiny \label};
    }
    
    % Highlight line 3090
    \draw[fill=green!50] (6,3) rectangle (9,3.4);
    \node at (7.5,3.2) {\tiny \textbf{Line 3090}};
    
\end{tikzpicture}
\end{center}

\section{Tổng kết}

\begin{tcolorbox}[colback=resultbox,title=Kết quả cuối cùng]
\textbf{Phần 1:}
\begin{center}
\begin{tabular}{|c|c|c|}
\hline
\textbf{Tag} & \textbf{Index} & \textbf{Block Offset} \\
\hline
14 bit & 12 bit & 6 bit \\
\hline
\end{tabular}
\end{center}

\textbf{Phần 2:} Với địa chỉ 180525:
\begin{itemize}
    \item \textbf{Vùng nhớ (Tag):} 2
    \item \textbf{Cache Line:} 3090
    \item \textbf{Block:} 2820
    \item \textbf{Byte trong block:} 45
\end{itemize}
\end{tcolorbox}

\end{document}
