\documentclass[12pt,a4paper]{article}
\usepackage{fontspec}
\usepackage[vietnamese]{babel}
\usepackage{amsmath,amssymb}
\usepackage{siunitx}
\usepackage{geometry}
\usepackage{xcolor}
\usepackage{tikz}
\usepackage{enumitem}
\usepackage{booktabs}
\usepackage{fancyhdr}
\usepackage{tcolorbox}

\geometry{margin=2cm}
\setlength{\headheight}{15pt}
\addtolength{\topmargin}{-3pt}
\pagestyle{fancy}
\fancyhf{}
\fancyhead[L]{EE5215 - Câu 2: Tần số CPU}
\fancyhead[R]{1.0 điểm}
\fancyfoot[C]{\thepage}

\definecolor{formulabox}{RGB}{230,240,255}
\definecolor{resultbox}{RGB}{220,255,220}
\definecolor{conceptbox}{RGB}{255,245,220}

\begin{document}

\begin{center}
\Large\textbf{CÂU HỎI 2: TẦN SỐ CPU VÀ HIỆU NĂNG}\\[0.3cm]
\large Learning Outcome: L.O.1 | Điểm: 1.0
\end{center}

\section{Đề bài}

Máy tính A hoạt động ở tần số 2GHz thực thi một chương trình trong 10s. Các kỹ sư muốn thiết kế một máy tính B có khả năng thực thi chương trình đó chỉ trong thời gian 7s bằng cách tăng tần số so với máy tính A. 

Tuy nhiên, khi tăng tần số thì số chu kỳ cần thiết để thực thi chương trình trên máy tính B cũng tăng lên \textbf{1.4 lần} so với số chu kỳ cần thiết để thực thi chương trình trên máy tính A. 

Hỏi tần số máy tính B phải là bao nhiêu để đạt được thời gian xử lý như mong muốn?

\section{Kiến thức nền tảng}

\subsection{Mối quan hệ giữa Tần số, Chu kỳ và Thời gian thực thi}

\begin{tcolorbox}[colback=conceptbox,title=Các khái niệm cơ bản]
\textbf{1. Tần số (Frequency - f):}
\begin{itemize}
    \item Đơn vị: Hz (Hertz) = số chu kỳ trong 1 giây
    \item 1 GHz = $10^9$ Hz = 1 tỷ chu kỳ/giây
\end{itemize}

\textbf{2. Chu kỳ clock (Clock Period - T):}
\begin{itemize}
    \item Thời gian của 1 chu kỳ xung nhịp
    \item $T = \frac{1}{f}$
    \item Ví dụ: $f = 2\,\text{GHz} \Rightarrow T = 0.5\,\text{ns}$
\end{itemize}

\textbf{3. Số chu kỳ thực thi (Clock Cycles - C):}
\begin{itemize}
    \item Tổng số chu kỳ cần để hoàn thành chương trình
    \item Phụ thuộc vào: số lệnh, CPI (Cycles Per Instruction), kiến trúc
\end{itemize}
\end{tcolorbox}

\subsection{Công thức CPU Performance}

\begin{tcolorbox}[colback=formulabox,title=Công thức quan trọng]
\textbf{Công thức chính:}
\[
\boxed{T_{execution} = C \times T_{cycle} = \frac{C}{f}}
\]

Trong đó:
\begin{itemize}
    \item $T_{execution}$: Thời gian thực thi chương trình (giây)
    \item $C$: Số chu kỳ clock cần thiết
    \item $T_{cycle}$: Thời gian 1 chu kỳ (giây)
    \item $f$: Tần số CPU (Hz)
\end{itemize}

\textbf{Suy ra:}
\[
C = T_{execution} \times f
\]
\[
f = \frac{C}{T_{execution}}
\]
\end{tcolorbox}

\subsection{Tại sao tăng tần số lại tăng số chu kỳ?}

\begin{tcolorbox}[colback=conceptbox,title=Giải thích hiện tượng]
Khi tăng tần số CPU, một số yếu tố làm tăng số chu kỳ:

\begin{enumerate}
    \item \textbf{Độ trễ bộ nhớ tăng (Memory Stalls):}
    \begin{itemize}
        \item CPU nhanh hơn $\Rightarrow$ đợi bộ nhớ lâu hơn (relative)
        \item Cache miss penalty tính bằng số chu kỳ tăng lên
    \end{itemize}
    
    \item \textbf{Pipeline hazards tăng:}
    \begin{itemize}
        \item Data hazards, control hazards gây stall
        \item Số chu kỳ stall tăng khi clock nhanh hơn
    \end{itemize}
    
    \item \textbf{Thiết kế đơn giản hơn:}
    \begin{itemize}
        \item Để đạt tần số cao, có thể cần chia pipeline thành nhiều tầng hơn
        \item Nhiều tầng $\Rightarrow$ nhiều chu kỳ hơn cho mỗi lệnh
    \end{itemize}
\end{enumerate}
\end{tcolorbox}

\section{Lời giải chi tiết}

\subsection{Bước 1: Tổng hợp thông tin đã cho}

\begin{center}
\begin{tabular}{|l|c|c|}
\hline
\textbf{Thông số} & \textbf{Máy A} & \textbf{Máy B} \\
\hline
Tần số & $f_A = 2\,\text{GHz}$ & $f_B = ?$ \\
\hline
Thời gian thực thi & $T_A = 10\,\text{s}$ & $T_B = 7\,\text{s}$ \\
\hline
Số chu kỳ & $C_A$ & $C_B = 1.4 \times C_A$ \\
\hline
\end{tabular}
\end{center}

\subsection{Bước 2: Tính số chu kỳ của máy A ($C_A$)}

Áp dụng công thức: $C = T_{execution} \times f$

\begin{tcolorbox}[colback=formulabox]
\[
C_A = T_A \times f_A
\]
\[
C_A = 10\,\text{s} \times 2 \times 10^9\,\text{Hz}
\]
\[
C_A = 20 \times 10^9 \text{ chu kỳ} = \mathbf{20 \text{ tỷ chu kỳ}}
\]
\end{tcolorbox}

\textbf{Ý nghĩa:} Chương trình cần 20 tỷ chu kỳ clock để hoàn thành trên máy A.

\subsection{Bước 3: Tính số chu kỳ của máy B ($C_B$)}

Theo đề bài, số chu kỳ máy B tăng 1.4 lần so với máy A:

\begin{tcolorbox}[colback=formulabox]
\[
C_B = 1.4 \times C_A
\]
\[
C_B = 1.4 \times 20 \times 10^9
\]
\[
C_B = 28 \times 10^9 \text{ chu kỳ} = \mathbf{28 \text{ tỷ chu kỳ}}
\]
\end{tcolorbox}

\textbf{Ý nghĩa:} Cùng một chương trình nhưng trên máy B (tần số cao hơn), cần thêm 8 tỷ chu kỳ do các stall từ memory và pipeline.

\subsection{Bước 4: Tính tần số máy B ($f_B$)}

Từ công thức: $f = \frac{C}{T_{execution}}$

\begin{tcolorbox}[colback=resultbox,title=Kết quả cuối cùng]
\[
f_B = \frac{C_B}{T_B}
\]
\[
f_B = \frac{28 \times 10^9}{7}
\]
\[
f_B = 4 \times 10^9\,\text{Hz} = \boxed{\mathbf{4\,\text{GHz}}}
\]
\end{tcolorbox}

\section{Kiểm tra lại kết quả}

\begin{tcolorbox}[colback=white,title=Verification]
\textbf{Với $f_B = 4$ GHz, kiểm tra thời gian thực thi:}
\[
T_B = \frac{C_B}{f_B} = \frac{28 \times 10^9}{4 \times 10^9} = 7\,\text{s} \quad \checkmark
\]

\textbf{So sánh tỷ lệ tần số:}
\[
\frac{f_B}{f_A} = \frac{4}{2} = 2
\]
Máy B có tần số gấp đôi máy A.

\textbf{So sánh tỷ lệ thời gian:}
\[
\frac{T_A}{T_B} = \frac{10}{7} = 1.43
\]
Máy B nhanh hơn 1.43 lần (không phải 2 lần như tỷ lệ tần số vì số chu kỳ tăng).
\end{tcolorbox}

\section{Phân tích thêm}

\subsection{Hiệu quả tăng tần số}

\begin{tcolorbox}[colback=conceptbox]
\textbf{Speed-up thực tế vs. Speed-up lý tưởng:}

\textbf{Lý tưởng} (nếu số chu kỳ không đổi):
\[
\text{Speed-up}_{ideal} = \frac{f_B}{f_A} = \frac{4}{2} = 2\times
\]
Nghĩa là nếu số chu kỳ bằng nhau, máy B sẽ chạy nhanh gấp 2 lần.

\textbf{Thực tế} (do số chu kỳ tăng 1.4×):
\[
\text{Speed-up}_{actual} = \frac{T_A}{T_B} = \frac{10}{7} = 1.43\times
\]

\textbf{Efficiency (Hiệu quả):}
\[
\text{Efficiency} = \frac{\text{Speed-up}_{actual}}{\text{Speed-up}_{ideal}} = \frac{1.43}{2} = 0.715 = 71.5\%
\]

``Mất'' 28.5\% hiệu năng do overhead khi tăng tần số.
\end{tcolorbox}

\subsection{Công thức tổng quát}

Nếu $C_B = k \times C_A$ (với $k > 1$ là hệ số tăng chu kỳ):

\begin{tcolorbox}[colback=formulabox]
\[
f_B = \frac{C_B}{T_B} = \frac{k \times C_A}{T_B} = \frac{k \times T_A \times f_A}{T_B}
\]

Trong bài này: $k = 1.4$, $T_A = 10$, $T_B = 7$, $f_A = 2$:
\[
f_B = \frac{1.4 \times 10 \times 2}{7} = \frac{28}{7} = 4\,\text{GHz}
\]
\end{tcolorbox}

\section{Tổng kết}

\begin{center}
\begin{tabular}{|l|c|c|}
\hline
\textbf{Thông số} & \textbf{Máy A} & \textbf{Máy B} \\
\hline
Tần số & 2 GHz & \textbf{4 GHz} \\
\hline
Thời gian thực thi & 10 s & 7 s \\
\hline
Số chu kỳ & 20 tỷ & 28 tỷ (+40\%) \\
\hline
Speed-up so với A & 1× & 1.43× \\
\hline
\end{tabular}
\end{center}

\textbf{Kết luận:} Để giảm thời gian thực thi từ 10s xuống 7s trong khi số chu kỳ tăng 1.4 lần, máy tính B cần có tần số \textbf{4 GHz}, gấp đôi máy A.

\end{document}
