\documentclass[12pt,a4paper]{article}
\usepackage{fontspec}
\usepackage[vietnamese]{babel}
\usepackage{amsmath,amssymb}
\usepackage{siunitx}
\usepackage{geometry}
\usepackage{xcolor}
\usepackage{tikz}
\usetikzlibrary{circuits.ee.IEC,shapes,arrows,positioning}
\usepackage{enumitem}
\usepackage{booktabs}
\usepackage{fancyhdr}
\usepackage{tcolorbox}
\usepackage{multirow}
\usepackage{circuitikz}

\geometry{margin=2cm}
\pagestyle{fancy}
\fancyhf{}
\fancyhead[L]{EE5215 - Câu 4: Dynamic Power}
\fancyhead[R]{1.5 điểm}
\fancyfoot[C]{\thepage}

\definecolor{formulabox}{RGB}{230,240,255}
\definecolor{resultbox}{RGB}{220,255,220}
\definecolor{conceptbox}{RGB}{255,245,220}
\definecolor{warningbox}{RGB}{255,230,230}

\begin{document}

\begin{center}
\Large\textbf{CÂU HỎI 4: DYNAMIC POWER}\\[0.3cm]
\large Learning Outcome: L.O.3 | Điểm: 1.5
\end{center}

\section{Đề bài}

\begin{enumerate}
    \item Dynamic power là gì (hình vẽ minh họa, giải thích, công thức tính)?
    \item Trình bày chi tiết (về hình, công thức, giải thích) 2 kỹ thuật dùng để giảm Dynamic power?
    \item Nhận xét ưu-khuyết điểm của 2 kỹ thuật đó.
\end{enumerate}

\section{Phần 1: Dynamic Power là gì?}

\subsection{Tổng quan về Power trong CMOS}

\begin{tcolorbox}[colback=conceptbox,title=Các thành phần công suất tiêu thụ]
Tổng công suất tiêu thụ của chip CMOS:
\[
P_{total} = P_{dynamic} + P_{static} + P_{short-circuit}
\]

\begin{itemize}
    \item \textbf{Dynamic Power ($P_{dynamic}$):} Công suất khi mạch chuyển trạng thái (switching)
    \item \textbf{Static Power ($P_{static}$):} Công suất rò rỉ (leakage) khi mạch không hoạt động
    \item \textbf{Short-circuit Power:} Công suất ngắn mạch trong quá trình chuyển đổi
\end{itemize}

Với công nghệ cũ (> 90nm): Dynamic power chiếm chủ đạo (70-90\%)\\
Với công nghệ mới (< 45nm): Static power ngày càng quan trọng
\end{tcolorbox}

\subsection{Định nghĩa Dynamic Power}

\begin{tcolorbox}[colback=formulabox,title=Định nghĩa]
\textbf{Dynamic Power} (Công suất động) là công suất tiêu thụ khi các transistor trong mạch số \textbf{chuyển trạng thái} từ 0$\rightarrow$1 hoặc 1$\rightarrow$0.

Công suất này phát sinh do quá trình \textbf{nạp và xả điện tụ tải} ($C_L$) thông qua các transistor.
\end{tcolorbox}

\subsection{Mạch CMOS Inverter - Minh họa}

\begin{center}
\begin{circuitikz}[scale=0.9]
    % VDD
    \draw (0,4) node[above]{$V_{DD}$} to[short] (0,3.5);
    
    % PMOS
    \draw (0,3.5) to[Tpmos, n=pmos] (0,2);
    \node at (-1.5,2.75) {PMOS};
    
    % NMOS
    \draw (0,2) to[Tnmos, n=nmos] (0,0.5);
    \node at (-1.5,1.25) {NMOS};
    
    % Ground
    \draw (0,0.5) to[short] (0,0) node[ground]{};
    
    % Input
    \draw (-2,1.75) node[left]{$V_{in}$} to[short] (-0.7,1.75);
    \draw (-0.7,1.75) to[short] (-0.7,2.75);
    \draw (-0.7,1.75) to[short] (-0.7,1.25);
    
    % Output
    \draw (0,2) to[short] (1,2);
    \draw (1,2) to[C, l=$C_L$] (1,0);
    \draw (1,0) to[short] (0,0);
    \draw (1,2) to[short] (2,2) node[right]{$V_{out}$};
    
    % Labels
    \node at (3.5,3) {\textbf{Khi $V_{in} = 0$:}};
    \node at (5,2.5) {PMOS ON, NMOS OFF};
    \node at (5,2) {$V_{out} = V_{DD}$ (HIGH)};
    \node at (5,1.5) {$C_L$ được nạp điện};
    
    \node at (3.5,0.5) {\textbf{Khi $V_{in} = V_{DD}$:}};
    \node at (5,0) {PMOS OFF, NMOS ON};
    \node at (5,-0.5) {$V_{out} = 0$ (LOW)};
    \node at (5,-1) {$C_L$ xả điện qua GND};
\end{circuitikz}
\end{center}

\subsection{Quá trình tiêu thụ năng lượng}

\begin{tcolorbox}[colback=white,title=Phân tích năng lượng]
\textbf{Chuyển từ 0 $\rightarrow$ 1 (Rising edge):}
\begin{itemize}
    \item PMOS dẫn, nạp điện cho $C_L$ từ $V_{DD}$
    \item Năng lượng lấy từ nguồn: $E_{supply} = C_L \cdot V_{DD}^2$
    \item Năng lượng lưu trong tụ: $E_{stored} = \frac{1}{2} C_L \cdot V_{DD}^2$
    \item Năng lượng tiêu tán trên PMOS: $E_{dissipated} = \frac{1}{2} C_L \cdot V_{DD}^2$
\end{itemize}

\textbf{Chuyển từ 1 $\rightarrow$ 0 (Falling edge):}
\begin{itemize}
    \item NMOS dẫn, xả điện từ $C_L$ xuống GND
    \item Năng lượng lưu trong tụ được xả: $E_{discharged} = \frac{1}{2} C_L \cdot V_{DD}^2$
    \item Năng lượng tiêu tán trên NMOS: $\frac{1}{2} C_L \cdot V_{DD}^2$
\end{itemize}

\textbf{Tổng năng lượng cho 1 chu kỳ đầy đủ (0$\rightarrow$1$\rightarrow$0):}
\[
E_{cycle} = C_L \cdot V_{DD}^2
\]
\end{tcolorbox}

\subsection{Công thức Dynamic Power}

\begin{tcolorbox}[colback=resultbox,title=Công thức chính]
\[
\boxed{P_{dynamic} = \alpha \cdot C_L \cdot V_{DD}^2 \cdot f}
\]

Trong đó:
\begin{itemize}
    \item $\alpha$ = \textbf{Activity Factor} (Hệ số hoạt động)
    \begin{itemize}
        \item Xác suất node chuyển trạng thái trong 1 chu kỳ
        \item Giá trị từ 0 đến 1
        \item Ví dụ: Clock có $\alpha = 1$ (chuyển mỗi chu kỳ)
        \item Data path thường có $\alpha = 0.1 \sim 0.3$
    \end{itemize}
    
    \item $C_L$ = \textbf{Load Capacitance} (Điện dung tải)
    \begin{itemize}
        \item Bao gồm: capacitance của wire, gate input, drain junction
        \item Đơn vị: Farad (thường là fF hoặc pF)
    \end{itemize}
    
    \item $V_{DD}$ = \textbf{Supply Voltage} (Điện áp nguồn)
    \begin{itemize}
        \item Điện áp cung cấp cho chip
        \item Đơn vị: Volt
    \end{itemize}
    
    \item $f$ = \textbf{Clock Frequency} (Tần số xung nhịp)
    \begin{itemize}
        \item Số chu kỳ clock trong 1 giây
        \item Đơn vị: Hz
    \end{itemize}
\end{itemize}
\end{tcolorbox}

\subsection{Ví dụ tính toán}

\begin{tcolorbox}[colback=formulabox]
\textbf{Cho:} Chip có $10^7$ transistors, mỗi transistor có $C_L = 10$ fF, $\alpha_{avg} = 0.1$, $V_{DD} = 1.2$V, $f = 3$ GHz.

\textbf{Tính:}
\begin{align*}
P_{dynamic} &= \alpha \cdot C_{total} \cdot V_{DD}^2 \cdot f \\
&= 0.1 \times (10^7 \times 10 \times 10^{-15}) \times (1.2)^2 \times (3 \times 10^9) \\
&= 0.1 \times 10^{-7} \times 1.44 \times 3 \times 10^9 \\
&= 43.2 \text{ W}
\end{align*}
\end{tcolorbox}

\section{Phần 2: Các kỹ thuật giảm Dynamic Power}

\subsection{Kỹ thuật 1: Voltage Scaling (Giảm điện áp nguồn)}

\begin{tcolorbox}[colback=conceptbox,title=Nguyên lý Voltage Scaling]
Vì $P_{dynamic} \propto V_{DD}^2$, việc giảm điện áp nguồn sẽ giảm công suất \textbf{theo bình phương}.

\textbf{Nếu giảm $V_{DD}$ xuống còn $k$ lần ($0 < k < 1$):}
\[
P_{new} = \alpha \cdot C_L \cdot (k \cdot V_{DD})^2 \cdot f = k^2 \cdot P_{old}
\]

\textbf{Ví dụ:} Giảm $V_{DD}$ từ 1.2V xuống 0.9V (tỷ lệ $k = 0.75$):
\[
\frac{P_{new}}{P_{old}} = (0.75)^2 = 0.5625 \quad \Rightarrow \text{Giảm 43.75\% công suất!}
\]
\end{tcolorbox}

\subsubsection{Dynamic Voltage and Frequency Scaling (DVFS)}

\begin{center}
\begin{tikzpicture}[scale=0.9]
    \draw[->] (0,0) -- (8,0) node[right] {Workload};
    \draw[->] (0,0) -- (0,5) node[above] {Power/Voltage};
    
    % High performance
    \draw[fill=red!30] (0.5,4) rectangle (2.5,4.5);
    \node at (1.5,4.25) {\tiny High Perf};
    \node at (1.5,3.5) {$V_{DD} = 1.2V$};
    \node at (1.5,3) {$f = 3$ GHz};
    
    % Medium
    \draw[fill=yellow!30] (3,2.5) rectangle (5,3);
    \node at (4,2.75) {\tiny Medium};
    \node at (4,2) {$V_{DD} = 1.0V$};
    \node at (4,1.5) {$f = 2$ GHz};
    
    % Low power
    \draw[fill=green!30] (5.5,1) rectangle (7.5,1.5);
    \node at (6.5,1.25) {\tiny Low Power};
    \node at (6.5,0.6) {$V_{DD} = 0.8V$};
    \node at (6.5,0.2) {$f = 1$ GHz};
\end{tikzpicture}
\end{center}

\subsubsection{Ưu điểm và Nhược điểm}

\begin{center}
\begin{tabular}{|p{7cm}|p{7cm}|}
\hline
\textbf{Ưu điểm} & \textbf{Nhược điểm} \\
\hline
\begin{itemize}[leftmargin=*]
    \item Hiệu quả rất cao (giảm theo $V^2$)
    \item Có thể điều chỉnh động (DVFS)
    \item Được hỗ trợ rộng rãi trong CPU/SoC hiện đại
    \item Giảm cả dynamic và short-circuit power
\end{itemize}
&
\begin{itemize}[leftmargin=*]
    \item Giảm tốc độ hoạt động (delay tăng)
    \item Có giới hạn dưới ($V_{DD} > V_{th}$)
    \item Giảm noise margin
    \item Cần voltage regulator hiệu quả cao
    \item Thời gian chuyển đổi voltage
\end{itemize}
\\
\hline
\end{tabular}
\end{center}

\subsubsection{Quan hệ Voltage - Delay}

\begin{tcolorbox}[colback=warningbox,title=Trade-off quan trọng]
Khi giảm $V_{DD}$, delay tăng theo công thức (gần đúng):
\[
t_{delay} \propto \frac{C_L \cdot V_{DD}}{(V_{DD} - V_{th})^2}
\]

Để duy trì timing correctness, phải \textbf{giảm tần số} tương ứng:
\[
f_{max} \propto \frac{(V_{DD} - V_{th})^2}{V_{DD}}
\]
\end{tcolorbox}

\subsection{Kỹ thuật 2: Clock Gating (Khóa xung nhịp)}

\begin{tcolorbox}[colback=conceptbox,title=Nguyên lý Clock Gating]
Ngắt xung clock cho các module/block không hoạt động để loại bỏ switching activity.

Khi clock bị tắt: $\alpha = 0 \Rightarrow P_{dynamic} = 0$ cho phần đó.

\textbf{Ước tính:} Clock distribution network có thể tiêu thụ 30-50\% tổng dynamic power. Clock gating có thể giảm 30-60\% power.
\end{tcolorbox}

\subsubsection{Sơ đồ Clock Gating}

\begin{center}
\begin{tikzpicture}[scale=0.8]
    % Without clock gating
    \node at (-4,3) {\textbf{Không có Clock Gating:}};
    \draw (-5,2) -- (-3,2) node[right] {CLK};
    \draw[fill=blue!20] (-3,1) rectangle (-1,2.5);
    \node at (-2,1.75) {Register};
    \draw (-1,1.75) -- (0,1.75);
    \node at (-2,0.5) {\textcolor{red}{Clock luôn chạy}};
    \node at (-2,0) {\textcolor{red}{dù không cần}};
    
    % With clock gating
    \node at (4,3) {\textbf{Có Clock Gating:}};
    \draw (1,2) -- (2,2);
    \node at (1.5,2.3) {\tiny CLK};
    
    % AND gate for clock gating
    \draw (2,1.5) -- (2,2.5) -- (3,2) -- cycle;
    \node at (2.5,2) {\tiny AND};
    \draw (1.5,1.2) -- (2,1.7);
    \node at (1.2,1.2) {\tiny EN};
    
    \draw (3,2) -- (4,2);
    \draw[fill=blue!20] (4,1) rectangle (6,2.5);
    \node at (5,1.75) {Register};
    \draw (6,1.75) -- (7,1.75);
    
    \node at (5,0.5) {\textcolor{green!60!black}{Clock chỉ active}};
    \node at (5,0) {\textcolor{green!60!black}{khi EN = 1}};
\end{tikzpicture}
\end{center}

\subsubsection{Clock Gating Cell (ICG)}

\begin{tcolorbox}[colback=formulabox]
\textbf{Integrated Clock Gating Cell} bao gồm:
\begin{itemize}
    \item Latch để giữ enable signal ổn định
    \item AND gate để kết hợp clock với enable
    \item Thiết kế đặc biệt để tránh glitch
\end{itemize}

\begin{center}
\begin{tikzpicture}[scale=0.7]
    \draw (0,0) rectangle (4,3);
    \node at (2,2.5) {\textbf{ICG Cell}};
    
    % Inputs
    \draw (-1,2) -- (0,2) node[left,xshift=-1cm] {CLK};
    \draw (-1,1) -- (0,1) node[left,xshift=-1cm] {EN};
    
    % Latch inside
    \draw[fill=yellow!30] (0.5,0.5) rectangle (2,1.5);
    \node at (1.25,1) {\tiny Latch};
    
    % AND inside
    \draw (2.5,1) -- (2.5,2.5) -- (3.5,1.75) -- cycle;
    \node at (3,1.75) {\tiny AND};
    
    % Output
    \draw (4,1.75) -- (5,1.75) node[right] {GCLK};
\end{tikzpicture}
\end{center}
\end{tcolorbox}

\subsubsection{Ưu điểm và Nhược điểm}

\begin{center}
\begin{tabular}{|p{7cm}|p{7cm}|}
\hline
\textbf{Ưu điểm} & \textbf{Nhược điểm} \\
\hline
\begin{itemize}[leftmargin=*]
    \item Không ảnh hưởng đến performance khi hoạt động
    \item Dễ triển khai trong RTL design
    \item Có thể giảm 30-60\% dynamic power
    \item Tự động được EDA tools hỗ trợ
    \item Có thể áp dụng ở nhiều mức (module, register)
\end{itemize}
&
\begin{itemize}[leftmargin=*]
    \item Thêm logic (ICG cells) = thêm diện tích, power
    \item Có thể gây clock skew
    \item Tăng độ phức tạp timing analysis
    \item Cần xác định đúng điều kiện enable
    \item Verification phức tạp hơn
\end{itemize}
\\
\hline
\end{tabular}
\end{center}

\section{Phần 3: So sánh hai kỹ thuật}

\begin{tcolorbox}[colback=resultbox,title=Bảng so sánh tổng hợp]
\begin{center}
\begin{tabular}{|l|c|c|}
\hline
\textbf{Tiêu chí} & \textbf{Voltage Scaling} & \textbf{Clock Gating} \\
\hline
\hline
Hiệu quả giảm power & Rất cao ($\propto V^2$) & Cao (30-60\%) \\
\hline
Ảnh hưởng performance & Giảm tốc độ & Không ảnh hưởng \\
\hline
Độ phức tạp thiết kế & Trung bình & Thấp-Trung bình \\
\hline
Overhead diện tích & Voltage regulator & ICG cells \\
\hline
Granularity & Toàn chip/domain & Module/Register \\
\hline
Dynamic adjustment & Có (DVFS) & Có (runtime enable) \\
\hline
Tool support & Tốt & Rất tốt \\
\hline
Verification effort & Cao & Trung bình \\
\hline
\end{tabular}
\end{center}
\end{tcolorbox}

\subsection{Khi nào nên dùng kỹ thuật nào?}

\begin{tcolorbox}[colback=conceptbox]
\textbf{Dùng Voltage Scaling khi:}
\begin{itemize}
    \item Có thể chấp nhận giảm performance
    \item Workload thay đổi theo thời gian (DVFS)
    \item Cần giảm power tối đa
    \item Battery-powered devices
\end{itemize}

\textbf{Dùng Clock Gating khi:}
\begin{itemize}
    \item Không thể chấp nhận giảm performance
    \item Có các modules hoạt động không thường xuyên
    \item Thiết kế có nhiều peripheral blocks
    \item Cần cân bằng power/performance
\end{itemize}

\textbf{Kết hợp cả hai:} Thực tế, các SoC hiện đại sử dụng \textbf{nhiều kỹ thuật kết hợp} bao gồm cả Voltage Scaling, Clock Gating, và Power Gating để tối ưu power ở mọi điều kiện hoạt động.
\end{tcolorbox}

\end{document}
